%%%%%%%%%%%%%%%%%%%% Documentación del sistema %%%%%%%%%%%%%%%%%%%%


\section{Documentación de sistema}

Documentación referida a los aspectos del análisis, diseño, implementación y prueba del \textit{software}, así como la implantación del mismo. En general, debe hacer referencia al sistema, ya que está orientada a los programadores que van a realizar el mantenimiento. Debe estar muy bien estructurada, desde lo más general a los más interno.   


\subsection{Especificación del sistema}

Esta sección debe describir el análisis y la especificación de requisitos. Cómo se descompone el problema en distintos subproblemas y los módulos asociados a cada uno de ellos, acompañados de su especificación. También debe incluir el plan de desarrollo del \textit{software}.  


\subsection{Módulos}

Debe contener una descripción de cada módulo, su funcionalidad y la interfaz. También se podría añadir código fuente.  


\subsection{Plan de prueba}

Debe describir todo el plan de pueba que se ha llevado a cabo. Se distinguen las pruebas realizadas a cada módulo y las pruebas de integración de los distintos módulos probándolo como un todo. Por último, también se debe incluir la descripción del plan de pruebas de aceptación.  


\subsubsection{Prueba de los módulos}

Incluir toda la batería de pruebas realizadas. Pruebas de caja blanca y pruebas de caja negra.  


\subsubsection{Prueba de integración}

Incluir toda la batería de pruebas realizadas. Pruebas de caja blanca y pruebas de caja negra.  


\subsubsection{ Plan de pruebas de aceptación}

Estas pruebas deben diseñarse con el usuario del sistema. Deben describir las pruebas que son necesarias pasar para que el sistema sea aceptado por el usuario final.  

\bigskip

Si en algún punto del documento se quiere hacer referencia a algún documento es necesario incluir la cita donde corresponda, podemos tomar como ejemplo \cite{Ejemplo3}. 