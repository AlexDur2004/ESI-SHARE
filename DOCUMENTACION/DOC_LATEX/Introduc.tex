%%%%%%%%%%%%%%%%%%%% Introducción %%%%%%%%%%%%%%%%%%%%


\section{Introducción}

\begin{center}
    \textbf{¿Eres parte de la comunidad universitaria de la Escuela Superior de Ingeniería y buscas una alternativa de transporte sostenible y económica?}\\
\end{center}
\begin{center}
    \textbf{¿Te gustaría compartir tu vehículo con otros miembros de la comunidad que recorran la misma ruta?}\\
\end{center}
Si es así, estás de suerte, ya que hemos desarrollado un programa para compartir coche, que es exclusivo para los miembros de la comunidad de la ESI.

\bigskip

¿Cómo funciona? Es fácil, ESI-SHARE cuenta con una interfaz amigable e intuitiva que facilita la gestión de los viajes compartidos.  Los usuarios pueden registrar sus
viajes y especificar la ruta y horario, así como el número de plazas disponibles en su vehículo, para otros miembros de la comunidad que compartan la misma ruta.
También, pueden buscar viajes disponibles que se ajusten a sus necesidades y reservarlos. Todo esto con el objetivo de reducir el número de vehículos en circulación,
y disminuir el impacto ambiental del transporte.

\bigskip

En definitiva, esta es una herramienta innovadora que contribuye al fomento de la movilidad sostenible y al cuidado del medio ambiente, a la vez que
promueve la colaboración y el uso eficiente de los recursos.
\bigskip
\begin{center}
    \textbf{¿Te animas a probarlo?}
\end{center}