%%%%%%%%%%%%%%%%%%%% Documentación del usuario %%%%%%%%%%%%%%%%%%%%


\section{Documentación de usuario}

Documentación relacionada con las funciones del sistema pero no se hace referencia a nada relativo a su implementación. Está orientado a las personas que van a usar el sistema no a los que lo van a mantener.  


\subsection{Descripción funcional}

\begin{itemize}
  \item Debe describir de forma simple el propósito del sistema.
  \item En esta sección se debe incluir una breve descripción funcional sobre lo que hace y no hace el sistema.
  \item Se pueden incluir pequeños ejemplos para dar una visión general de las características del sistema.  
\end{itemize}


\subsection{Tecnología}

Un listado con una breve descripción de las tecnologías usadas en el proyecto: aplicaciones, frameworks, librerías, arquitecturas,...
Por ejemplo, se puede añadir información sobre el tipo de programa, desarrollador, última versión estable y enlace a la página web oficial, entre otros detalles. 
También se pueden añadir los iconos representativos de las herramientas.

\bigskip

Si se necesita hacer referencia a algunos documentos se incluyen las citas que sean necesarias \citep{Ejemplo1, Ejemplo2}.


\subsection{Manual de instalación}

Una breve descripción que explique como instalar el sistema y adecuarlo a las configuraciones particulares del \textit{hardware}, indicando las características mínimas de \textit{hardware} requeridas para el funcionamiento del programa.  


\subsection{Acceso al sistema}

Debe contener una explicación sencilla de cómo iniciarse en el sistema y cómo salir del mismo. Así como la forma de salir de los problemas más habituales.  


\subsection{Manual de referencia}

En esta sección se describe con detalle las ventajas que ofrece el sistema a los usuarios y cómo se pueden obtener mediante su uso. Debe describir, de una manera más completa y formal, el uso del sistema, así como las situaciones de error y los informes generados.  


\subsection{Guía del operador}

Si el programa requiriera de operador entonces se incluiría esta sección. Debe describir cómo se debe de reaccionar antes determinadas situaciones que pueden darse durante el uso del sistema.  

